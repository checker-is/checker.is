\documentclass[handout]{beamer}
%\documentclass{beamer}

\usepackage{multirow,array}

%\usepackage[utf8]{inputenc}

%\usepackage[T1]{fontenc}
%\usepackage{mathabx}
%\usepackage{mathpazo}
%\usepackage{eulervm}
\usepackage{textpos}
%\usepackage{verbatim}
\usepackage{hyperref}
\usepackage{listings}

\usepackage{textpos}
%\usepackage{verbatim}
\usepackage{hyperref}
\usepackage{minitoc}
\usepackage{fancyvrb}


\usetheme{Warsaw}

\setbeamercolor{normal text}{fg=white,bg=black!90}
\setbeamercolor{structure}{fg=white}

\setbeamercolor{alerted text}{fg=red!85!black}

\setbeamercolor{item projected}{use=item,fg=black,bg=item.fg!35}

\setbeamercolor*{palette primary}{use=structure,fg=structure.fg}
\setbeamercolor*{palette secondary}{use=structure,fg=structure.fg!95!black}
\setbeamercolor*{palette tertiary}{use=structure,fg=structure.fg!90!black}
\setbeamercolor*{palette quaternary}{use=structure,fg=structure.fg!95!black,bg=black!80}

\setbeamercolor*{framesubtitle}{fg=white}

\setbeamercolor*{block title}{parent=structure,bg=black!60}
\setbeamercolor*{block body}{fg=black,bg=black!10}
\setbeamercolor*{block title alerted}{parent=alerted text,bg=black!15}
\setbeamercolor*{block title example}{parent=example text,bg=black!15}

\def\signed #1{{\leavevmode\unskip\nobreak\hfil\penalty50\hskip2em
  \hbox{}\nobreak\hfil(#1)%
  \parfillskip=0pt \finalhyphendemerits=0 \endgraf}}

\newsavebox\mybox
\newenvironment{aquote}[1]{\savebox\mybox{#1}\begin{quote}}{\signed{\usebox\mybox}\end{quote}}

\newcommand{\yel}[
  1]{{\color{yellow} #1}}


%\setbeamertemplate{headline}[text line]{%
%some text}
%\setbeamertemplate{navigation symbols}{}


%\setbeamertemplate{itemize items}[default]
%\setbeamertemplate{enumerate items}[default]

\useinnertheme{rectangles}

\usefonttheme{serif}
\usecolortheme{rose}

\usepackage{fontspec}
\setmainfont[AutoFakeSlant=0.3]{Code2000}


\title{Checker}
\author[@arthurb]{Arthur Breitman\\@arthurb}
%\institute{Overleaf}

\begin{document}


\maketitle

%\frame{\tableofcontents}

%\setbeamertemplate{section in toc}[default]

\AtBeginSection[]{%
  \begin{frame}<beamer>
    \frametitle{Outline}
    \tableofcontents[sectionstyle=show/hide,subsectionstyle=hide/show/hide]
  \end{frame}
  \addtocounter{framenumber}{-1}% If you don't want them to affect the slide number
}

%\AtBeginSection[]{%
%  \begin{frame}
%    \frametitle{Outline}
%    \tableofcontents[sectionstyle=show/hide,subsectionstyle=hide/show/hide]
%  \end{frame}
%  \addtocounter{framenumber}{-1}% If you don't want them to affect the slide number
%}

\section{Design goal}

\begin{frame}{Disclaimer}
  \begin{itemize}
  \item These slides explore ideas for the design of a software \yel{technology}
  \item Like all exploratory ideas these are subject to change
  \item I may use \yel{loose vocabulary} and \yel{metaphors} to better convey ideas
  \item These \yel{metaphors} break down past a certain point
  \item The system should be understood based on what the actions its actually
    programmed to take
  \end{itemize}
\end{frame}

\begin{frame}{What is Checker?}
  \yel{Checker} is a \yel{software} technology for stabilizing a token with respect to an externally provided index.
  Concretely:
  \begin{itemize}
  \item Defined as a Tezos smart-contract
  \item Index can be any peg, e.g.
    \begin{itemize}
    \item Price of 1 oz t of gold in tez
    \item Price of 1 HKD in tez
    \item Viewcount of baby shark videos on January 1st 2030
    \end{itemize}
  \end{itemize}  
\end{frame}

\begin{frame}{Goals}
  Primariy design goal is to \yel{minimize trust}:
  \begin{itemize}
  \item Minimize discretionary inputs: minimalistic governance
  \item Minimize reliance on centralized point of failures: no centralized collateral
  \item Minimize reliance on external actors
  \end{itemize}
  Design \yel{limits}:
  \begin{itemize}
  \item Peg can break
  \item Parameter tuning is still worth having
  \item Oracle feeds are a weak point
  \end{itemize}
\end{frame}

\section{Design}

\begin{frame}{Model Design}
  \begin{itemize}
  \item \yel{Modular} design to make it easier to reason about the system.
  \item Relies on three different tokens
    \begin{itemize}
    \item Draught token
    \item Checker token
    \item \yel{Kits}
    \end{itemize}    
  \end{itemize}
  \end{frame}

\section{Draught}

\begin{frame}{Locking tez}
  Locking tez in Draught. User selects:
  \begin{itemize}
  \item Locking \yel{time} (e.g 1 day, 1 month, 1 year)
  \item Unlocking \yel{amount} (e.g. spend 100 kit to unlock)
  \item Forfeit \yel{threshold} (e.g. 120 kit worth of tez)
  \end{itemize}
  Draught contract offers some amount of kits to the user
  based on the above variables and an internal \yel{model}.
  \begin{itemize}
  \item Draught contract sends kits (fewer than the unlocking amount)
  \item Tez is locked. \emph{User still controls delegation}.
  \end{itemize}
\end{frame}

\begin{frame}{Draught}
  Users locking tez interact with a \yel{Draught} contract
  \begin{itemize}
  \item Draught model determined by tunable parameters
  \item Draught is itself \yel{tokenized} in draught tokens
  \item Draught targets fixed kit holding, e.g. 5\%
    \begin{itemize}
    \item Cannot create or destroy kit
    \item Maintain enough kit to interact with users
    \item Create draught tokens and auction for kit
    \item Or auction kit for draught and burn
    \end{itemize}
  \item If user forfeits tez, tez held by Draught is auctionned off for kit
  \end{itemize}
\end{frame}

\section{Checker}

\begin{frame}{Checker}
  Checker is also a tokenized contract. Tokens have a dual purpose.
  \begin{itemize}
    \item \yel{Governance}
      \begin{itemize}
      \item Used to vote on Draught parameters
      \item Used to select an oracle feed
      \end{itemize}
    \item \yel{Stabilization}
      \begin{itemize}
      \item If kit above beg, \yel{mint kit}, auction off for checker tokens, burn
      \item If kit below beg, \yel{mint checker tokens}, auction off for kit, burn
      \item \emph{Acquire draught tokens\ldots}
      \end{itemize}
  \end{itemize}      
\end{frame}

\begin{frame}{Checker and draught}
  Would this stabilization mechanism work by itself? Sort of \ldots
  \begin{itemize}
    \item Checker holders may want to minimize dilution
    \item Rational course of action:\\
      mint checker tokens, exchange for draught tokens
    \item Don't rely on business acumen, automate the process
  \end{itemize}
  How to automate?
  \begin{itemize}
  \item Draught exports \yel{kit unlock amount} per token
  \item Checker automatically mints checker tokens and auctions them off for draught tokens
  \item Checker has no privileged access to Draught \yel{tokens}!
  \end{itemize}
\end{frame}

\section{Misc}

\begin{frame}{Auctions}
  How does a smart-contract exchange one kind of token for another? Auctions
  \begin{itemize}
  \item \yel{Simple} to analyze
  \item Descending auction is \yel{gas sparing}
  \item Typical horizon on the order of days
  \item Alternatives (e.g. Dexter) assume liquidity
  \item Downside: slow
  \end{itemize}
\end{frame}

\begin{frame}{Oracle feed}
  Contract needs two oracle feed
  \begin{itemize}
  \item \yel{tez} with respect to peg
  \item \yel{kit} with respect to tez
  \end{itemize}
  A few remarks
  \begin{itemize}
  \item Second one observable on-chain
  \item Oracle construction is out of scope for Checker
    \begin{itemize}
    \item Median of 5 reputable sources
    \item Decentralized Oracle à la Augur
    \end{itemize}
  \end{itemize}
\end{frame}

\begin{frame}{What about MakerDAO?}
  \begin{itemize}
  \item Similar to MakerDAO with a few key differences as outlined above
  \item Pretty much the only reasonable category of design for this
  \item Dai has survived very wild periods relatively unscathed
  \end{itemize}
\end{frame}

\begin{frame}{Failure scenarios}
  Here are a few ways in which this can break down
  \begin{itemize}
  \item Malicious \yel{mispricing} from oracle feed 
  \item Forfeitures \yel{spiral}
  \item Demand for kit can outstrip interest for locking tez\\
    \yel{this is the weird / important one}
  \item Draught token cornered
  \item Checker governance picking loose Draught parameters
  \item Implementation \yel{bugs}
  \end{itemize}
\end{frame}

\begin{frame}{Questions}
  Questions ?
\end{frame}
  


  
\end{document}
